\documentclass{article}

\usepackage{geometry}
\usepackage{blindtext}
\usepackage{tabularx}
\usepackage{float}
\usepackage{titlesec}
\usepackage{verbatim}
\usepackage{amsmath}

\geometry{
    a4paper,
    left=15mm,
    right=15mm,
    top=15mm,
    bottom=15mm
}

\pagestyle{headings}

\begin{document}
    \begin{titlepage}
        \title{Common Application Settings Interface}
        \author{Studievereniging Sticky, MrFriendly B.V.}
        \date{2024\\Version 1.0.0}

        \maketitle
        \tableofcontents

        \pagebreak
    \end{titlepage}

    \section{Goal}
    \label{sec:goal}

    This specification describes a format for applications to retrieve
    settings from other applications in a standardized way.
    This allows applications to communicate with each other, and more importently,
    configure themselves to suit other applications.

    \subsection{Example: Scope managment}
    \label{subsec:example-scope-managment}

    A good example of when such a common interface is useful is
    authorization.
    An OAuth2 server can authorize a client to use specific scopes.
    But, to know about the scopes of the applications behind the OAuth2 application
    would require manual configuration.
    A standardized way of retrieving the list of available scopes would help in this situation.

    \subsection{Example: Settings page}
    \label{subsec:example-settings-page}

    Another good application is a common settings page.
    An application usually has some configuration parameters.
    One could build a settings page per application.
    But, better would be one settings page for all application.
    This specification describes methods for retrieving and setting such
    settings, allowing for one common page for everything.

    \section{Components}
    \label{sec:components}

    The common application settings interface (CASI) has multiple components to it:
    \begin{enumerate}
        \item General application information
        \item Supported scopes
        \item Application settings
        \item Deletion of user data (AVG)
    \end{enumerate}

    \section{General application information}
    \label{sec:general-application-information}
    The general application information interface is an endpoint
returning a JSON document containing information about the application.
This interface may be at any HTTP path.
This path is known as the CASI path.

\subsection{Request}
\label{subsec:request-general-application-information}

A HTTP GET request should be executed.
The path for this endpoint may be determined by the application itself.

\subsection{Response}
\label{subsec:response-general-application-information}

\begin{itemize}
    \item The $case\_version$ value must be the version of the CASI specification the application is compliant with.
    \item The $application\_version$ value must be semver\footnote{https://doc.rust-lang.org/cargo/reference/semver.html} comptabile.
\end{itemize}

\verbatiminput{responses/200_ok}
\verbatiminput{responses/general_application_information.json}

    \section{Supported scopes}
    \label{sec:supported-scopes}
    This endpoint returns a document showing which authorization scopes are supported
by the application.

\subsection{Request}
\label{subsec:request-supported-scopes}

A HTTP GET request should be executed.
The path for this endpoint is configured in the general application infromation document.

\subsection{Response}
\label{subsec:response-supported-scopes}

\verbatiminput{responses/200_ok}
\verbatiminput{responses/supported_scopes.json}


    \section{Application settings}
    \label{sec:application-settings}
    This set of endpoints allows other applications to configure application settings

\subsection{List of available settings}
\label{subsec:list-of-available-settings}

\subsubsection{Request}
A HTTP GET request should be executed.
This endpoint is protect using an OAuth2 access token.
The path for this endpoint is configured in the general application infromation document.

\verbatiminput{requests/get_available_settings}

\subsubsection{Response}

The $type$ value is one of:
\begin{equation}
    \label{eq:response-list-of-available-settings}
    \text{Type} = \{uint32, int32, uint64, int64, bool, String\}
\end{equation}
Types can be annoted with $?$, e.g.\ $uint32?$, can be null.

\verbatiminput{responses/200_ok}
\verbatiminput{responses/available_settings.json}

\subsection{Get settings value}
\label{subsec:get-settings-value}

\subsubsection{Request}

A HTTP GET request should be executed.
This endpoint is protect using an OAuth2 access token.
The $key$ query parameter should be one of the available settings returned in section\ \ref{subsec:list-of-available-settings}

\verbatiminput{requests/get_settings_value}

\subsubsection{Response}

\verbatiminput{responses/200_ok}
\verbatiminput{responses/get_settings_value.json}

\subsection{Set settings value}
\label{subsec:set-settings-value}

\subsubsection{Request}

A HTTP POST request should be executed.
This endpoint is protect using an OAuth2 access token.

\verbatiminput{requests/post_with_authentication}
\verbatiminput{requests/set_settings_value.json}

\subsubsection{Response}

\verbatiminput{responses/200_ok}
\verbatiminput{responses/set_settings_value.json}

    \section{Deletion of user data}
    \label{sec:deletion-of-user-data}
    This endpoint is used to delete the data of a specific user, making an application easily AVG compliant.

\subsection{Request}
\label{subsec:request-delete-user-data}

A HTTP POST request should be executed.
This endpoint is protect using an OAuth2 access token.

The application must ensure that the authenticated user is:
\begin{itemize}
    \item deleting themselves, should always be allowed
    \item deleting another user, should only be allowed for admins
\end{itemize}

\verbatiminput{requests/post_with_authentication}
\verbatiminput{requests/delete_user_data.json}

\subsection{Response}
\label{subsec:response-delete-user-data}

\verbatiminput{responses/200_ok}
\verbatiminput{responses/delete_user_data.json}

\end{document}